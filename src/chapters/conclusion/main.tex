\chapter{Conclusion} % (fold)
\label{cha:conclusion}

In this dissertation, it was designed and implemented a tracking library for the web called \textit{tracking.js}. This work aims to provide a common infrastructure to develop applications and to accelerate the use of those techniques on the web in commercial products.

In a second moment, several basic concepts were introduced for a better understanding of the tracking library solution proposed in this work.

Later on, a description of \textit{tracking.js} were presented, such as the library concepts, explaining its base and advanced functionalities and modules. Three main techniques were explored, a Markerless Tracking Algorithm in the Section \ref{sec:tracking_library_for_the_web:marker_less_tracking_algorithm}, a Rapid Object Detection (Viola Jones) in the Section \ref{sec:tracking_library_for_the_web:rapid_object_detection} and a Color Tracking Algorithm in the Section \ref{sec:tracking_library_for_the_web:color_tracking_algorithm}.

In the evaluation of the library, were presented for each available library technique a description, results, performance metrics and discussions. Some real-world applications were prototyped in order to test the library running on the web browser.

For Rapid Object Detection, it was presented different training data being used by the library implementation of Viola Jones, and a simple chat application was created. In this chat application, while talking in real-time, the users could augment their faces with objects, such as a fake glass with mustache.

For the Color Tracking Algorithm, it was presented its usage for different colored objects, such as a red pencil marker, a Rubik’s magic cube \cite{Rubiks2013} and a red Ball of Whacks \cite{Whack2013}. Also few examples were implemented using color tracking and a PlayStation move controller, such as a multi-player game that allows the user to draw using the camera, controlling a 3D environment trough user interactions and an application that the user can control the volume of the browser audio player sliding a colored object in front of the camera.

For the Markerless Tracking Algorithm, it was demonstrated that the user is able to track any object based on invariant natural features characteristics. Then, benefits of a JavaScript tracking solution are presented bellow.

All the proposed modules run on native web browsers without requiring third-party plugins installation, therefore any browser-ready device can eventually use the proposed cross-platform code base to develop AR applications. The possibility to use \textit{tracking.js} as a cross-platform library is a reality. The browser environment is evolving fast and due to the cross-platform ability, JavaScript is becoming a popular solution for multiple devices and platforms. In a near future, other devices and visual displays may, potentially, embed browser versions as well and they could all benefit from \textit{tracking.js} library.

\section{Contributions} % (fold)
\label{sec:conclusion:contributions}

There are several contributions of this work. First is to provide an academic material about web browser concepts and API and how they apply to visual tracking and AR applications. Second is to pioneering a library that brings state-of-the-art techniques for visual tracking, computer vision and AR solutions on the web browser without requiring any third-party plugin installation. Third is based on several optimizations for existing techniques in order to allow them to run with real-time rate on the web browser. The first optimization is how the Rapid Object Detection technique   merges found rectangles, defined in Section \ref{sec:tracking_library_for_the_web:rapid_object_detection}. This work proposes the replacement of the disjoint set data structure with an alternative logic that is called ``Minimum Neighbor Area Grouping'' by this dissertation. Minimum Neighbor Area Grouping has $O(N^2)$ performance \cite{black2007big} and consists in a loop trough the possible rectangle faces returned by the scanning detector. The second optimization is a fast and robust color blob detection proposed in the color tracking technique defined in Section \ref{sec:tracking_library_for_the_web:color_tracking_algorithm}.

% section contributions (end)

\section{Future work} % (fold)
\label{sec:conclusion:future_work}

The web browser is an environment that still doesn't have the appropriate attention inside the academic world. This research has raised many possibilities. The first one is a publication detailing web browser concepts and APIs that can leverage visual tracking and AR applications to be developed into this platform. The second is another publication comparing client-side JavaScript based tracking with client-side plugin-based and server-side solutions. The growth of the library modules is also part of future work plan, trough bringing more related algorithms and techniques to \textit{tracking.js} library, such as Double Exponential Smoothing \cite{LaViola2003}, a Perspective-$n$-Point problem (P$n$P) calculation and a image processing layer capable of simple transformations and filters, such as saturation, sharpen, sobel, laplacian, gaussian and prewitt \cite{Gonzalez2007}. Those improvements could potentially help visual tracking and AR techniques to be implemented on top of \textit{tracking.js}.

% section future_work (end)

% chapter conclusion (end)