\chapter{Conclusion} % (fold)
\label{cha:conclusion}

Lorem ipsum dolor sit amet, consectetur adipisicing elit.

\section{Contributions} % (fold)
\label{sec:conclusion:contributions}

There are several contributions of this work. First is to provide an academic material about web browser concepts and API and how they apply to visual tracking and AR applications. Second is to pioneering a library that brings state-of-the-art techniques for visual tracking, computer vision and AR solutions on the web browser without requiring any third-party plugin installation. Third is based on several optimizations for existing techniques in order to allow them to run with real-time rate on the web browser. The first optimization is how the Rapid Object Detection technique   merges found rectangles, defined in Section \ref{sec:tracking_library_for_the_web:rapid_object_detection}. This work proposes the replacement of the disjoint set data structure with an alternative logic that is called ``Minimum Neighbor Area Grouping'' by this dissertation. Minimum Neighbor Area Grouping has $O(N^2)$ performance \cite{black2007big} and consists in a loop trough the possible rectangle faces returned by the scanning detector. The second optimization is a fast and robust color blob detection proposed on the color tracking technique defined in Section \ref{sec:tracking_library_for_the_web:color_tracking_algorithm}.

% section contributions (end)

\section{Future Work} % (fold)
\label{sec:conclusion:future_work}

The web browser is an environment that still doesn't have the appropriate attention inside the academic world. This research has raised many possibilities. The first one is a publication detailing web browser concepts and APIs that can leverage visual tracking and AR applications to be developed into this platform. The second is another publication comparing client-side JavaScript based tracking with client-side plugin-based and server-side solutions. The growth of the library modules is also part of future work plan, trough bringing more related algorithms and techniques to \textit{tracking.js} library, such as Double Exponential Smoothing \cite{LaViola2003}, a Perspective-$n$-Point problem (P$n$P) calculation and a image processing layer capable of simple transformations and filters, such as saturation, sharpen, sobel, laplacian, gaussian and prewitt \cite{Gonzalez2007}. Those improvements could potentially help visual tracking and AR techniques to be implemented on top of \textit{tracking.js}.

% section future_work (end)

% chapter conclusion (end)