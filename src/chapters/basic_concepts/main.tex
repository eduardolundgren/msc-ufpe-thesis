\chapter{Basic Concepts} % (fold)
\label{cha:basic_concepts}

Lorem ipsum dolor sit amet, consectetur adipisicing elit, sed do eiusmod
tempor incididunt ut labore et dolore magna aliqua. Ut enim ad minim veniam,
quis nostrud exercitation ullamco laboris nisi ut aliquip ex ea commodo
consequat. Duis aute irure dolor in reprehenderit in voluptate velit esse
cillum dolore eu fugiat nulla pariatur. Excepteur sint occaecat cupidatat non
proident, sunt in culpa qui officia deserunt mollit anim id est laborum.

\section{Web} % (fold)
\label{sec:basic_concepts:web}

Using concepts from existing hypertext systems, Tim Berners-Lee, computer scientist and at that time employee of CERN, wrote a proposal in March 1989 for what would eventually become the World Wide Web (WWW) \cite{w3c:history}.

The World Wide Web is a shared information system operating on top of the Internet. Web browsers retrieve content and display from remote web servers using a stateless and anonymous protocol called HyperText Transfer Protocol (HTTP). Web pages are written using a simple language called HyperText Markup Language (HTML). They may be augmented with other technologies such as Cascading Style Sheets (CSS), which adds additional layout and style information to the page, and JavaScript language, which allows client-side computation. Browsers typically provide other useful features such as bookmarking, history, password management, and accessibility features to accommodate users with disabilities \cite{grosskurth2005reference}.

In the beginning of the web, plain text and images were the most advanced features available on the browsers. Companies behind web browser development together with the web community, were able to contribute to the World Wide Web Consortium (W3C) specifications \cite{w3c:www}. Today's web is a result of the ongoing efforts of an open web community that helps define these technologies and ensure that they're supported in all web browsers. Those contributions transformed the web in a growing universe of interlinked pages and applications, with videos, photos, interactive content, 3D graphics processed by the Graphics Processing Unit (GPU), and other varieties of features without requiring any third-party plugins installation.

There are five major browsers used today: Internet Explorer, Firefox, Safari, Chrome and Opera. Currently, the usage share of Firefox, Safari and Chrome together is nearly 60\%. The browser main functionality is to present a web resource, by requesting it from the server and displaying it on the browser window. The resource is usually a HTML document.

* Contextualization
    * Problems of augmented reality on the web
* State of the art
    * History of web
    * W3C
    * Browsers
        * The browser's high level structure
        * The browser's main functionality
    * HTML5
    * JavaScript
        * Language details
        * Typed arrays
        * requestAnimationFrame
        * getUserMedia
    * Canvas
    * Video
    * WebRTC
    * APIs

    \cite{w3c:history}.

\subsection{State of the Art} % (fold)
\label{sub:basic_concepts:web:state_of_the_art}

% subsection state_of_the_art (end)

\subsection{Problems of Augmented Reality on the Web} % (fold)
\label{sub:basic_concepts:web:problems_of_augmented_reality_on_the_web}

% subsection problems_of_augmented_reality_on_the_web (end)

% section web (end)

\section{Augmented Reality} % (fold)
\label{sec:basic_concepts:augmented_reality}

Lorem ipsum dolor sit amet, consectetur adipisicing elit, sed do eiusmod
tempor incididunt ut labore et dolore magna aliqua. Ut enim ad minim veniam,
quis nostrud exercitation ullamco laboris nisi ut aliquip ex ea commodo
consequat. Duis aute irure dolor in reprehenderit in voluptate velit esse
cillum dolore eu fugiat nulla pariatur. Excepteur sint occaecat cupidatat non
proident, sunt in culpa qui officia deserunt mollit anim id est laborum.

\subsection{State of the Art} % (fold)
\label{sub:basic_concepts:augmented_reality:state_of_the_art}

Lorem ipsum dolor sit amet, consectetur adipisicing elit, sed do eiusmod
tempor incididunt ut labore et dolore magna aliqua. Ut enim ad minim veniam,
quis nostrud exercitation ullamco laboris nisi ut aliquip ex ea commodo
consequat. Duis aute irure dolor in reprehenderit in voluptate velit esse
cillum dolore eu fugiat nulla pariatur. Excepteur sint occaecat cupidatat non
proident, sunt in culpa qui officia deserunt mollit anim id est laborum.

% subsection state_of_the_art (end)

% section augmented_reality (end)

\section{Tracking and Object Detection} % (fold)
\label{sec:basic_concepts:tracking}

Lorem ipsum dolor sit amet, consectetur adipisicing elit, sed do eiusmod
tempor incididunt ut labore et dolore magna aliqua. Ut enim ad minim veniam,
quis nostrud exercitation ullamco laboris nisi ut aliquip ex ea commodo
consequat. Duis aute irure dolor in reprehenderit in voluptate velit esse
cillum dolore eu fugiat nulla pariatur. Excepteur sint occaecat cupidatat non
proident, sunt in culpa qui officia deserunt mollit anim id est laborum.

\subsection{State of the Art} % (fold)
\label{sub:basic_concepts:tracking:state_of_the_art}

% subsection state_of_the_art (end)

% section tracking (end)

% chapter basic_concepts (end)