\title{Tracking Library for the Web}

\author{Eduardo A. Lundgren Melo}
% If you wish to list your previous degrees on the cover page, use the
% previous degrees command:
%       \prevdegrees{A.A., Harvard University (1985)}
% You can use the \\ command to list multiple previous degrees
%       \prevdegrees{B.S., University of California (1978) \\
%                    S.M., Massachusetts Institute of Technology (1981)}
\department{\CIn}

\degree{Master of Science in Computer Science}
\degreemonth{August}
\degreeyear{2013}
\thesisdate{Aug 30, 2013}

%% By default, the dissertation will be copyrighted to CIn. If you need to copyright
%% the dissertation to yourself, just specify the `vi' documentclass option.  If for
%% some reason you want to exactly specify the copyright notice text, you can
%% use the \copyrightnoticetext command.
%\copyrightnoticetext{\copyright IBM, 2013.  Do not open till Xmas.}

% If there is more than one supervisor, use the \supervisor command
% once for each.
\supervisor{Silvio de Barros Melo}{Advisor}
\supervisor{Veronica Teichrieb}{Co-Advisor}

% This is the department committee chairman, not the dissertation committee
% chairman.  You should replace this with your Department's Committee
% Chairman.
\chairman{\ufpe}{Department Committee on Master Theses}

% Make the titlepage based on the above information.  If you need
% something special and can't use the standard form, you can specify
% the exact text of the titlepage yourself.  Put it in a titlepage
% environment and leave blank lines where you want vertical space.
% The spaces will be adjusted to fill the entire page.  The dotted
% lines for the signatures are made with the \signature command.
% \maketitle

% The abstractpage environment sets up everything on the page except
% the text itself.  The title and other header material are put at the
% top of the page, and the supervisors are listed at the bottom.  A
% new page is begun both before and after.  Of course, an abstract may
% be more than one page itself.  If you need more control over the
% format of the page, you can use the abstract environment, which puts
% the word "Abstract" at the beginning and single spaces its text.

%% You can either \input (*not* \include) your abstract file, or you can put
%% the text of the abstract directly between the \begin{abstractpage} and
%% \end{abstractpage} commands.

% First copy: start a new page, and save the page number.
\cleardoublepage
% Uncomment the next line if you do NOT want a page number on your
% abstract and acknowledgments pages.
% \pagestyle{empty}
\setcounter{savepage}{\thepage}
\begin{abstractpage}
%% The text of your abstract and nothing else (other than comments) goes here.
%% It will be single-spaced and the rest of the text that is supposed to go on
%% the abstract page will be generated by the abstractpage environment.  This
%% file should be \input (not \include 'd) from cover.tex.
In this thesis, I designed and implemented an Augmented Reality (AR) framework
to provide a common infrastructure for AR applications and to accelerate the use
of those techniques on the web in commercial products.   It runs on native web
browsers without requiring third-party plugins installation; This involves use
of several modern browser specifications as well as implementation
of different computer vision algorithms and techniques into the browser environment.
These algorithms can be used to detect and recognize faces, identify objects,
track moving objects, etc.  The source language of the framework is JavaScript
that is the language interpreted by all modern browsers.   The majority of
interpreted languages have limited computational power when compared to compiled
languages, such as C; The computational complexity involved in AR requires highly
optimized implementations in order to have good results.  A series of evaluation
tests were made, to determine how effective these thecniques were on the web.



\end{abstractpage}

\begin{abstractpage}
%% The text of your abstract and nothing else (other than comments) goes here.
%% It will be single-spaced and the rest of the text that is supposed to go on
%% the abstract page will be generated by the abstractpage environment. This
%% file should be \input (not \include 'd) from cover.tex.
Nesta dissertação, foi concebida e implementada uma biblioteca de visão computacional para navegadores web com o objetivo de fornecer uma infra-estrutura comum para desenvolver aplicativos e acelerar a utilização dessas técnicas na web em produtos comerciais. A biblioteca proposta tem como foco ser utilizada em navegadores web sem a necessidade de instalação de plugins de terceiros. Várias especificações web modernas foram utilizadas para alcançar o resultado esperado, bem como aplicação de diferentes algoritmos de visão computacional. A solução provê a implementação de algoritmos existentes que podem ser utilizados para diferentes aplicações nesta área, tais como, detecção de faces, identificação de objetos e cores, como também rastrear objetos em movimento. Os navegadores web modernos interpretam a linguagem de programação JavaScript, portanto esta foi a linguagem utilizada na base da biblioteca. A maioria das linguagens interpretadas têm limitado poder computacional quando comparado com linguagens compiladas, como C. A complexidade computacional envolvida em algoritmos de rastreamento de vídeos é alta e requer implementações otimizadas. Algumas otimizações são discutidas e implementadas neste trabalho, a fim de alcançar bons resultados quando comparados com implementações similares em linguagens compiladas. Uma série de testes de avaliação foram feitos para determinar a eficácia dessas técnicas na web.
\\\\
Palavras-chave: Ciência da Computação, Visão Computacional, Web
\end{abstractpage}

% Additional copy: start a new page, and reset the page number.  This way,
% the second copy of the abstract is not counted as separate pages.
% Uncomment the next 6 lines if you need two copies of the abstract
% page.
% \setcounter{page}{\thesavepage}
% \begin{abstractpage}
% %% The text of your abstract and nothing else (other than comments) goes here.
%% It will be single-spaced and the rest of the text that is supposed to go on
%% the abstract page will be generated by the abstractpage environment.  This
%% file should be \input (not \include 'd) from cover.tex.
In this thesis, I designed and implemented an Augmented Reality (AR) framework
to provide a common infrastructure for AR applications and to accelerate the use
of those techniques on the web in commercial products.   It runs on native web
browsers without requiring third-party plugins installation; This involves use
of several modern browser specifications as well as implementation
of different computer vision algorithms and techniques into the browser environment.
These algorithms can be used to detect and recognize faces, identify objects,
track moving objects, etc.  The source language of the framework is JavaScript
that is the language interpreted by all modern browsers.   The majority of
interpreted languages have limited computational power when compared to compiled
languages, such as C; The computational complexity involved in AR requires highly
optimized implementations in order to have good results.  A series of evaluation
tests were made, to determine how effective these thecniques were on the web.



% \end{abstractpage}

\cleardoublepage

\section*{Acknowledgments}

I would like to formally say thank you to all that somehow contributed to this master thesis and all the education I had before it. These two years of dedication will worth each second in the rest of my life. I thank to my advisor, and to my co-advisor for all the incentive in order to finish this work. Special thank you to all the amazing friends I have made during the master program, Thiago Rocha, Lucas Figueiredo, Francisco Magalhães, Silvio Melo and Veronica Teichrieb, I would like to let them know without their support and mentoring the quality of this work would be not as close as it ended up being. Also, I cannot forget to be thankful to Renata, my fiancé, to be so supportive all this time.