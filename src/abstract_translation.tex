%% The text of your abstract and nothing else (other than comments) goes here.
%% It will be single-spaced and the rest of the text that is supposed to go on
%% the abstract page will be generated by the abstractpage environment. This
%% file should be \input (not \include 'd) from cover.tex.
Nesta dissertação, foi concebida e implementada uma biblioteca de visão computacional para navegadores web com o objetivo de fornecer uma infra-estrutura comum para desenvolver aplicativos e acelerar a utilização dessas técnicas na web em produtos comerciais. A biblioteca proposta tem como foco ser utilizada em navegadores web sem a necessidade de instalação de plugins de terceiros. Várias especificações web modernas foram utilizadas para alcançar o resultado esperado, bem como aplicação de diferentes algoritmos de visão computacional. A solução provê a implementação de algoritmos existentes que podem ser utilizados para diferentes aplicações nesta área, tais como, detecção de faces, identificação de objetos e cores, como também rastrear objetos em movimento. Os navegadores web modernos interpretam a linguagem de programação JavaScript, portanto esta foi a linguagem utilizada na base da biblioteca. A maioria das linguagens interpretadas têm limitado poder computacional quando comparado com linguagens compiladas, como C. A complexidade computacional envolvida em algoritmos de rastreamento de vídeos é alta e requer implementações otimizadas. Algumas otimizações são discutidas e implementadas neste trabalho, a fim de alcançar bons resultados quando comparados com implementações similares em linguagens compiladas. Uma série de testes de avaliação foram feitos para determinar a eficácia dessas técnicas na web.